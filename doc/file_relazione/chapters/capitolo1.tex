\maketitle
\chapter {Intreduzione}
La traccia del progetto richiedeva agli studenti, sostanzialmente, di implementare l'algoritmo KNN (K Nearest Neighbors) utilizzato come modello di machine learning e impiegato per la classificazione degli oggetti.

L'obiettivo è quello di trovare, per ogni elemento della collezione Query i k punti più simili presenti nel dataset dei dati. 

La ricerca dei k elementi più simili viene fatta sulla base di una misurazione, la distanza euclidea, che va a calcolare la distanza fra vettori prendendo in considerazione ogni componente.

Il problema principale risiede nelle risorse che il calcolo della distanza euclidea, calcolata per ogni vettore presente in Q per tutti gli elementi presenti nel dataset, richiederebbe. Il tempo di esecuzione, sviluppando la distanza euclidea sarebbe $O(n*D)$ in cui n sono gli elementi del dataset e D sono